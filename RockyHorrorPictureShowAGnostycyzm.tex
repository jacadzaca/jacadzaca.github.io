\documentclass[12pt]{minimal}

\usepackage{polski}
\usepackage[utf8]{inputenc}
\usepackage[T1]{fontenc}
\usepackage{footmisc}
\usepackage{lmodern}

\author{jacadżaca}
\title{Rocky Horror Picture Show, a gnostycyzm}
\makeatletter
\begin{document}
\begin{center}
\begin{tabular}{c}
\textbf{\@author} \\
\textbf{\@title} \\
\textbf{\today}
\end{tabular}
\end{center}
Czy na po-północny seans \textit{double-feature} moglibyśmy się wybrać zamiast w
kabaretkach i gorsecie\footnote{fani \textit{Rocky Horror Picture Show} wytworzyli wokół 
wyświetleń filmów dziwny rytuał, mianowicie: mieli oni w zwyczaju przychodzić do kina, co
weekend, przebrani za swoje ulubione postacie, a do tego papugować kwestie swoich
ekranowych odpowiedników i śpiewać piosenki. Wydarzenia miały przypominać dziwną mszę.
Wstęp \textit{remake'u} jest ukłonem w stronę tej anegdotki} w szacie antycznego gnostyka?
Czy musical, który został stworzony w hołdzie przemijającej epoce horrorów i filmów
science-fiction klasy B zawiera w sobie pierwiastek gnozy? Nie, nie zawiera.
Jednakże, w mojej głowie wszystko jest możliwe, a \textit{Rocky Horror Picture Show} jest
nieznanym, ale zabawnym filmem z oryginalną muzyką i subtelną dozą samo-świadomości.
Czyż widmo krążące nad innymi interpretacjami, widmo trzymania się tekstu mogłoby mnie powstrzymać od wciągnięcia mojego opium?

Gnostycy wierzą, że ludzie dzielą się na dwa rodzaje: tych zagmatwanych w ziemskie sprawy,
żyjących w ignorancji boskiego pierwiastka w nich samych (hyletów) i pneumatyków,
zdolnych do duchowego poznania, żyjących np. w z zgodności z "poezją"\footnote{ po grecku
"poezja" oznacza: "Ja tworzę"; żyć w zgodności z poezją oznacza poświęcić swoje życie
tworzeniu; patrz hermetyzm}. Gnostycyzm nie głosi jednak teorii predestynacji,
pneumatykiem można zostać - szkopuł tkwi w tym, że jest to proces \emph{ezoteryczny},
wychodzący z wewnątrz.

W \textit{Rocky Horror Picture Show} wydajemy się być świadkami właśnie
takiego procesu. Brad i Janet na początku żyją w świecie materialnym - kiedy Brad
oświadcza się Janet, ją najbardziej ekscytuje nowy pierścionek, Brad całkowicie wypiera
nietypowość mieszkańców zamku. Wszystko się zmieni, kiedy w tajemniczych okolicznościach
trafiają do zamku z Transylwanii (w galaktyce Transexual), gdzie będzie im poznać to
\emph{coś} wewnątrz siebie.

Dr. Frank-N-Furter tworzy idealnego mężczyznę, który ma pomóc wyzwolić
mu "absolutną przyjemność"\footnote{ należy podkreślić, że nie jest to najbardziej
ortodoksyjne rozumienie boskiego pierwiastka}, czyli przemienić się w pneumatyka.
Brad i Janet także uczestniczą w tej iluminacji; obydwoje zdradzają drugie z Frankiem,
sami doznając, że w życiu jest \emph{coś} więcej. Procesy przemian kulminują się w  ostatnich
chwilach życia Franka. W piosence \textit{Rose Tint my World}, każdy z bohaterów
uzewnętrznia swoją iluminację, zwłaszcza Frank, który śpiewa, że w końcu jest tym, kim
marzył, żeby być (\textit{Don't dream it, be it}). W swoim ostatnim solo, Frank opowiada o
swojej gorzkiej realizacji, że aby na Ziemi nie jest jego miejscem, że musi opuścić tę 
niegościnną krainę (w końcu to świat stworzony przez Demiurga)\footnote{gnostycy uważają, 
że świat został w części stworzony przez złego boga, tj. Demiurga; zadaniem gnostyka jest
wyrwać się z jego zniewolenia i połączyć z platońskim światem, gdzie mieszka "prawdziwy"
Bóg}, musi zostawić piękne niebieskie niebo, na które patrzy przez łzy. Brad i Janet dają
wyraz swojemu bólowi w \textit{Super Heroes}, wypędzeni z zamku, gdzie mogli poczuć się
jednością z absolutem teraz "krwawią". Nawet Kryminolog to dostrzega, uogólnia; nazywa cały
rodzaj ludzki tylko insektami pełzającymi po Ziemi bez żadnego celu.

Mimo, że w interpretacji nie poruszyłem niektórych wątków (np. kim jest dr. Scott? dlaczego służący Frank-N-Furtera go zabijają?), to
\textit{Rocky Horror Picture Show} jest ciekawym artefaktem kultury, który ma w sobie aurę tajemniczości.
Z jednej strony jest to ukłon w stronę kiczowatego kina, czyli czegoś, co nie powinno być wartę uwagi, a z
drugiej filmem, w którym ludzie widzą to \emph{coś}, boski pierwiastek.
\end{document}

